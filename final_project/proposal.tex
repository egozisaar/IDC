\documentclass[11pt]{article}
\usepackage[margin=1in]{geometry}          
\usepackage{graphicx}
\usepackage{amsthm, amsmath, amssymb}
\usepackage{setspace}\onehalfspacing
\usepackage[loose,nice]{units}
\usepackage{array}
\usepackage[super]{nth}
\usepackage{graphicx}
\usepackage{float}
\usepackage{subcaption}
\usepackage{mathtools}
\usepackage[displaymath, mathlines]{lineno}
\usepackage{natbib}
\usepackage[all]{nowidow}
\usepackage{wrapfig}
\newenvironment{conditions}
  {\par\vspace{\abovedisplayskip}\noindent\begin{tabular}{>{$}l<{$} @{${}={}$} l}}
  {\end{tabular}\par\vspace{\belowdisplayskip}}
\newcommand{\R}{\mathbb{R}}
\usepackage{fancyhdr}
\pagestyle{fancy}
\fancyhf{}
\rhead{Saar Egozi: saartk@gmail.com}
\lhead{Final Project Proposal}
\cfoot{*Willing to present my work if requested}

\begin{document}
\section*{Baby Names Selection Models}
\subsection*{Introduction}
\paragraph{} The naming of a child may depend on several different reasons.
I want to know if one can model the frequency of a name in the population based solely on its frequency in the previous generation.
My goal is to try and create a model that successfully describes the naming phenomena as a selection model, similar to a biological selection model.
\paragraph{Hypothesis .} The naming phenomena can be modelled as a selection model, and a frequency-dependant model will fit the data better than a simple selection model with a constant selection bias.

\subsection*{Models \& Methods}
\paragraph{Model 1 .} Consider a population of size $N(t)$, where $t$ is the year.
Every baby born can be named either $A$ or any other name, which is categorised as $B$, where the relative fitnesses are: $w_A=1+s, w_B=1$, where $s > -1$ and is a constant.
\paragraph{Model 2 .} Consider a population of the same size as Model 1, with the same names types $A$ or $B$.
In this model $s_{t+1}=f(p_t)$, where $p_t$ is the frequency of the name $A$ in year $t$, and $s_{t+1}$ is the selection bias in year $t+1$ of $A$.
The relative fitnesses in year $t$ are therefore: $w_{t,A}=1+s_{t-1}, w_{t,B}=1$, where $s > -1$.
My initial selection coefficient function $f(p_t) = a\cdot p^2_t + b\cdot p_t + c$, where $a, b, c$ are the model's parameters.
During research I may use other frequency dependant functions that may better fit the data.
\paragraph{Model Fitting. } I intend to fit my models to the data using the \textbf{Maximum Likelihood Estimation} approach, using \textit{scipy} packages like \textit{optimize}.
I will then use model selection using the \textbf{F-test} approach.

\subsection*{Expected Results}
I expect to see a better fit of the frequency-dependant model to the data in more than 75\% of the names tested.
I will only fit names with frequency above a certain threshold, and with minimal amount of "noise" (i.e many sharp slope changes between years).




\end{document}

